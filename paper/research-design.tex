% Options for packages loaded elsewhere
\PassOptionsToPackage{unicode}{hyperref}
\PassOptionsToPackage{hyphens}{url}
\PassOptionsToPackage{dvipsnames,svgnames*,x11names*}{xcolor}
%
\documentclass[
  12pt,
]{article}
\usepackage{lmodern}
\usepackage{setspace}
\usepackage{amssymb,amsmath}
\usepackage{ifxetex,ifluatex}
\ifnum 0\ifxetex 1\fi\ifluatex 1\fi=0 % if pdftex
  \usepackage[T1]{fontenc}
  \usepackage[utf8]{inputenc}
  \usepackage{textcomp} % provide euro and other symbols
\else % if luatex or xetex
  \usepackage{unicode-math}
  \defaultfontfeatures{Scale=MatchLowercase}
  \defaultfontfeatures[\rmfamily]{Ligatures=TeX,Scale=1}
  \setmainfont[]{Times New Roman}
  \setsansfont[]{Times New Roman}
\fi
% Use upquote if available, for straight quotes in verbatim environments
\IfFileExists{upquote.sty}{\usepackage{upquote}}{}
\IfFileExists{microtype.sty}{% use microtype if available
  \usepackage[]{microtype}
  \UseMicrotypeSet[protrusion]{basicmath} % disable protrusion for tt fonts
}{}
\makeatletter
\@ifundefined{KOMAClassName}{% if non-KOMA class
  \IfFileExists{parskip.sty}{%
    \usepackage{parskip}
  }{% else
    \setlength{\parindent}{0pt}
    \setlength{\parskip}{6pt plus 2pt minus 1pt}}
}{% if KOMA class
  \KOMAoptions{parskip=half}}
\makeatother
\usepackage{xcolor}
\IfFileExists{xurl.sty}{\usepackage{xurl}}{} % add URL line breaks if available
\IfFileExists{bookmark.sty}{\usepackage{bookmark}}{\usepackage{hyperref}}
\hypersetup{
  colorlinks=true,
  linkcolor=Maroon,
  filecolor=Maroon,
  citecolor=Blue,
  urlcolor=Blue,
  pdfcreator={LaTeX via pandoc}}
\urlstyle{same} % disable monospaced font for URLs
\usepackage[margin=1in]{geometry}
\usepackage{longtable,booktabs}
% Correct order of tables after \paragraph or \subparagraph
\usepackage{etoolbox}
\makeatletter
\patchcmd\longtable{\par}{\if@noskipsec\mbox{}\fi\par}{}{}
\makeatother
% Allow footnotes in longtable head/foot
\IfFileExists{footnotehyper.sty}{\usepackage{footnotehyper}}{\usepackage{footnote}}
\makesavenoteenv{longtable}
\usepackage{graphicx,grffile}
\makeatletter
\def\maxwidth{\ifdim\Gin@nat@width>\linewidth\linewidth\else\Gin@nat@width\fi}
\def\maxheight{\ifdim\Gin@nat@height>\textheight\textheight\else\Gin@nat@height\fi}
\makeatother
% Scale images if necessary, so that they will not overflow the page
% margins by default, and it is still possible to overwrite the defaults
% using explicit options in \includegraphics[width, height, ...]{}
\setkeys{Gin}{width=\maxwidth,height=\maxheight,keepaspectratio}
% Set default figure placement to htbp
\makeatletter
\def\fps@figure{htbp}
\makeatother
\setlength{\emergencystretch}{3em} % prevent overfull lines
\providecommand{\tightlist}{%
  \setlength{\itemsep}{0pt}\setlength{\parskip}{0pt}}
\setcounter{secnumdepth}{5}

\title{Predicting emotional reactions to Facebook posts\\
by political actors\footnote{Term paper research design for the seminar ``Computational Social Science''}}
\author{Pirmin Stöckle\footnote{Corresponding address: \href{mailto:pirmin.stoeckle@gess.uni-mannheim.de}{\nolinkurl{pirmin.stoeckle@gess.uni-mannheim.de}}}\\
GESS and SFB 884, University of Mannheim}
\date{\today}

\begin{document}
\maketitle

\setstretch{1.2}
\hypertarget{introduction}{%
\section{Introduction}\label{introduction}}

Emotions play a central role in social life, of which politics is no exception. Political science research has investigated systematic effects of emotions on politically relevant behavior such as turnout (Valentino et al. \protect\hyperlink{ref-valentino2011election}{2011}), vote choice (Brader \protect\hyperlink{ref-brader2005striking}{2005}; Groenendyk \protect\hyperlink{ref-groenendyk2019two}{2019}), information search (Clifford and Jerit \protect\hyperlink{ref-clifford2018disgust}{2018}; Valentino et al. \protect\hyperlink{ref-valentino2008worried}{2008}), and public attitudes (Clifford \protect\hyperlink{ref-clifford2019how}{2019}).

It is certainly not a new idea that politcal actor face incentives to strategically employ emotions in their electoral campaigns -- and political communication more generally -- in order to gain public support. However, the study of emotions in political communication has received a recent push in interest due to methodological advancements, such as automated text analysis (Grimmer and Stewart \protect\hyperlink{ref-grimmer2013text}{2013}; Wilkerson and Casas \protect\hyperlink{ref-wilkerson2017large}{2017}) applied to sentiment and emotions in party manifestos (Crabtree et al. \protect\hyperlink{ref-crabtree2020how}{2020}), social media data (Widmann \protect\hyperlink{ref-widmann2021how}{2021}), and parliamentary speech (Osnabrügge, Hobolt, and Rodon \protect\hyperlink{ref-osnabrugge2021playing}{2021}).\footnote{And even the application of physiological measures (Bakker and Schumacher \protect\hyperlink{ref-bakker2021hot}{2021}).}

These studies provide insights into not only what, but how political actors communicate. Results indicate, e.g., that incumbent parties use more positive sentiment than opposition parties (Crabtree et al. \protect\hyperlink{ref-crabtree2020how}{2020}) and populist parties use more negative emotional appeals (anger, fear, disgust, sadness) and less positive emotional appeals (joy, enthusiasm, pride, hope) (Widmann \protect\hyperlink{ref-widmann2021how}{2021}).

Especially on social media, emotions are important, e.g.~because false news on Twitter spread faster than true news, possibly because of invoked emotional responses (Vosoughi, Roy, and Aral \protect\hyperlink{ref-vosoughi2018spread}{2018}). There is also evidence of emotional contagion on Facebook, shown by the observation that users exposed to fewer positive posts tend to produce fewer positive posts themselves (Kramer, Guillory, and Hancock \protect\hyperlink{ref-kramer2014experimental}{2014}). At the same time, social media provide political actors with unprecedented means to communicate directly to large audiences (Stier et al. \protect\hyperlink{ref-stier2018election}{2018}). Still, we know relatively little about the effects of emotional communication by polticial actors and its trategic use.

\hypertarget{research-question}{%
\section{Research question}\label{research-question}}

Against this background, I investigate emotional reactions to social media posts from political actors. More specifically, I ask: Which characteristics explain the amount of specific emotional reactions a post gets?

Substantively, I argue that a systematic link between political communication and reactions to it is a necessary condition for political communciation to be used strategically.

I expect that sentiment in posts by political actors affect user's emotional reactions to these posts. The research question of this study closely follows a recent study by Eberl et al. (\protect\hyperlink{ref-eberl2020what}{2020}). Using data from Austrian political actors during the 2017 Austrian election, they find that the sentiment of a Facebook post relates to the number of ``Love'' and ``Angry'' reactions to that post.

I build on this study, but go beyond it in several aspects. First, I analyze data from Germany and will extend the timeframe to a complete legislative cycle instead of one election campaign. Second, my empirical approach is different because I apply an exploratory approach based on machine learning, instead of formulating hypotheses in a confirmatory framework. The focus is thus on predicting \(\hat{Y}\) (reactions) instead of quantifying the contribution from one individual \(\hat{\beta_i}*X_i\). Such an approach is kind of new for political science but clearly on the rise (Grimmer, Roberts, and Stewart \protect\hyperlink{ref-grimmer2021machine}{2021}).

While the focus is on prediction, I will also evaluate the effects of subtantively interesting factors, i.e., evaluate their contribution for predicting the outcome.

\hypertarget{data-measurement}{%
\section{Data \& Measurement}\label{data-measurement}}

I use data on Facebook posts by political actors obtained via CrowdTangle. CrowdTangle is a public insights tool, whose main intent was to monitor what content overperformed in terms of interactions (likes, shares, etc.) on Facebook and other social media platforms. In 2016, CrowdTangle was acquired by Facebook that now provides the service.

After receiving individual access for scientific purposes, CrowdTangle offers extensive functionality to download Facebook posts and related metadata from public pages, such as verified official pages or individual accounts with large numbers of followers. The availability of the data goes back until the very start of Facebook and thus offers a large amount of data. Besides the textual post content, the data also contains account information, and -- crucially for this project -- the number of interactions with the post (for a codebook see Garmur et al. (\protect\hyperlink{ref-garmur2019crowdtangle}{2019})). These interactions include the number of likes, shares and comments.

Since 2016, Facebook offers a set of distinct reactions to posts expressed by emojis, which may be interpreted as emotional reactions (``Love'', ``Angry'', ``Haha'', ``Wow'', ``Sad'', ``Care''). While some of these are quite ambiguous, recent research has used the ``Love'' and ``Angry'' reactions as proxies for the respective emotional response (Eberl et al. \protect\hyperlink{ref-eberl2020what}{2020}; Muraoka et al. \protect\hyperlink{ref-muraoka2021love}{2021}).

For now, I will restrict the analysis to Germany, but in principle, the approach will be scalable to other cases as well, since Facebook coverage is wide and behavior on the platform can be analyzed using a common metric (Muraoka et al. \protect\hyperlink{ref-muraoka2021love}{2021}).

For the first step, I will also restrict the analyis to posts from party accounts, as this already presents a large data source without the need to identify all relevant accounts from individual politcal candidates. In the next step, however, I will include all posts from individual candidates, which will then represent a more complex, nested data structure (posts nested within accounts, nested within parties).

\hypertarget{estimation-strategy}{%
\section{Estimation strategy}\label{estimation-strategy}}

I intend to build a machine learning model predicting emotional reactions to Facebook posts. As outcome measure I will either use the number of ``Love'' /``Angry'' reactions or its share on all reactions.

As inputs, I intend to use three broad types of input: (i) metadata about the post as obtained from CrowdTangle, (ii) data derived from the message text, such as sentiment or topic, and (iii) external data from the information environment such as public issue salience, economic conditions, opinion polls, or time until election day.

The exact model specifications are yet to be defined. Model performance will be investigated on test data, which will be left untouched during the model training phase (e.g.~James et al. (\protect\hyperlink{ref-james2013introduction}{2013})).

\hypertarget{references}{%
\section{References}\label{references}}

\linespread{1}

\hypertarget{refs}{}
\leavevmode\hypertarget{ref-bakker2021hot}{}%
Bakker, Bert N. and Gijs Schumacher. 2021. ``Hot Politics? Affective Responses to Political Rhetoric.'' \emph{American Political Science Review} 115(1):150--64.

\leavevmode\hypertarget{ref-brader2005striking}{}%
Brader, Ted. 2005. ``Striking a Responsive Chord: How Political Ads Motivate and Persuade Voters by Appealing to Emotions.'' \emph{American Journal of Political Science} 49(2):388--405.

\leavevmode\hypertarget{ref-clifford2019how}{}%
Clifford, Scott. 2019. ``How Emotional Frames Moralize and Polarize Political Attitudes.'' \emph{Political Psychology} 40(1):75--91.

\leavevmode\hypertarget{ref-clifford2018disgust}{}%
Clifford, Scott and Jennifer Jerit. 2018. ``Disgust, Anxiety, and Political Learning in the Face of Threat.'' \emph{American Journal of Political Science} 62(2):266--79.

\leavevmode\hypertarget{ref-crabtree2020how}{}%
Crabtree, Charles, Matt Golder, Thomas Gschwend, and Indridi H. Indridason. 2020. ``It Is Not Only What You Say, It Is Also How You Say It: The Strategic Use of Campaign Sentiment.'' \emph{The Journal of Politics} 82(3):1044--60.

\leavevmode\hypertarget{ref-eberl2020what}{}%
Eberl, Jakob-Moritz, Petro Tolochko, Pablo Jost, Tobias Heidenreich, and Hajo G. Boomgaarden. 2020. ``What's in a post? How sentiment and issue salience affect users' emotional reactions on Facebook.'' \emph{Journal of Information Technology \& Politics} 17(1):48--65.

\leavevmode\hypertarget{ref-garmur2019crowdtangle}{}%
Garmur, M., Gary King, Z. Mukerjee, N. Persily, and B. Silverman. 2019. ``CrowdTangle platform and API: Codebook.''

\leavevmode\hypertarget{ref-grimmer2021machine}{}%
Grimmer, Justin, Margaret E. Roberts, and Brandon M. Stewart. 2021. ``Machine Learning for Social Science: An Agnostic Approach.'' \emph{Annual Review of Political Science} 24(1):395--419.

\leavevmode\hypertarget{ref-grimmer2013text}{}%
Grimmer, Justin and Brandon M. Stewart. 2013. ``Text as Data: The Promise and Pitfalls of Automatic Content Analysis Methods for Political Texts.'' \emph{Political Analysis} 21(3):267--97.

\leavevmode\hypertarget{ref-groenendyk2019two}{}%
Groenendyk, Eric. 2019. ``Of Two Minds, But One Heart: A Good `Gut' Feeling Moderates the Effect of Ambivalence on Attitude Formation and Turnout.'' \emph{American Journal of Political Science} 63(2):368--84.

\leavevmode\hypertarget{ref-james2013introduction}{}%
James, G., Daniela Witten, Trevor Hastie, and R. Tibshirani. 2013. \emph{An introduction to statistical learning}.

\leavevmode\hypertarget{ref-kramer2014experimental}{}%
Kramer, Adam D. I., Jamie E. Guillory, and Jeffrey T. Hancock. 2014. ``Experimental evidence of massive-scale emotional contagion through social networks.'' \emph{Proceedings of the National Academy of Sciences of the United States of America} 111(24):8788--90.

\leavevmode\hypertarget{ref-muraoka2021love}{}%
Muraoka, Taishi, Jacob Montgomery, Christopher Lucas, and Margit Tavits. 2021. ``Love and Anger in Global Party Politics.'' \emph{Journal of Quantitative Description: Digital Media} 1.

\leavevmode\hypertarget{ref-osnabrugge2021playing}{}%
Osnabrügge, Moritz, Sara B. Hobolt, and Toni Rodon. 2021. ``Playing to the Gallery: Emotive Rhetoric in Parliaments.'' \emph{American Political Science Review} 1--15.

\leavevmode\hypertarget{ref-stier2018election}{}%
Stier, Sebastian, Arnim Bleier, Haiko Lietz, and Markus Strohmaier. 2018. ``Election Campaigning on Social Media: Politicians, Audiences, and the Mediation of Political Communication on Facebook and Twitter.'' \emph{Political Communication} 35(1):50--74.

\leavevmode\hypertarget{ref-valentino2011election}{}%
Valentino, Nicholas A., Ted Brader, Eric W. Groenendyk, Krysha Gregorowicz, and Vincent L. Hutchings. 2011. ``Election Night's Alright for Fighting: The Role of Emotions in Political Participation.'' \emph{The Journal of Politics} 73(1):156--70.

\leavevmode\hypertarget{ref-valentino2008worried}{}%
Valentino, Nicholas A., Vincent L. Hutchings, Antoine J. Banks, and Anne K. Davis. 2008. ``Is a Worried Citizen a Good Citizen? Emotions, Political Information Seeking, and Learning via the Internet.'' \emph{Political Psychology} 29(2):247--73.

\leavevmode\hypertarget{ref-vosoughi2018spread}{}%
Vosoughi, Soroush, Deb Roy, and Sinan Aral. 2018. ``The spread of true and false news online.'' \emph{Science} 359(6380):1146--51.

\leavevmode\hypertarget{ref-widmann2021how}{}%
Widmann, Tobias. 2021. ``How Emotional Are Populists Really? Factors Explaining Emotional Appeals in the Communication of Political Parties.'' \emph{Political Psychology} 42(1):163--81.

\leavevmode\hypertarget{ref-wilkerson2017large}{}%
Wilkerson, John and Andreu Casas. 2017. ``Large-Scale Computerized Text Analysis in Political Science: Opportunities and Challenges.'' \emph{Annual Review of Political Science} 20(1):529--44.

\end{document}
